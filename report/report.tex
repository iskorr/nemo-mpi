\documentclass[12pt]{report}
  \title{\LARGE Distributed Simulation of Spiking Neural Networks using Message Passing Interface \\ \vspace{7mm} \large Final Year Project}
  \author{ \emph{Author:} \\ Iskander Orazbekov \vspace{5 mm} \\ \emph{Supervisor:} \\ Dr Timothy Todman \vspace{5 mm} \\ \emph{Second Marker:} \\ Professor Murray Shanahan \vspace{56 mm} \\ Imperial College London}
  \date{\today}
\usepackage{a4}
\usepackage{listings}
\lstset{
	language=C++,
	frame=single,
	numbers=left,
	breaklines=true,
	breakautoindent=true,
	xleftmargin={\ifodd\value{page}1em\else-\marginparwidth\fi},
	xrightmargin={\ifodd\value{page}-\marginparwidth\else1em\fi},
	basicstyle=\ttfamily\footnotesize\bfseries,
	frame=tb
}
\usepackage{graphicx}
\usepackage{url}
\usepackage[english]{babel}
\usepackage{wrapfig}
\usepackage[toc,page]{appendix}
\setlength{\parskip}{0.3cm}
\setlength{\parindent}{0cm}

\begin{document}

\maketitle

\begin{abstract}

Motivation behind the development of spiking neural networks arised from the growing interest in the area of neural 
computation and studies of brain activity in the recent years. The reason lies in the fact that SNNs incorporate time
into neuron firing computation and signify the role of interconnectedness between neurons, therefore bringing the simulation 
level closer to reality.

However, as a result of such improvement, spiking neural networks are computationally expensive to simulate, therefore, a solution 
using parallel computation was needed. NeMo, a Spiking Neural Network simulator developed at Imperial College, solves this problem 
by making use of CUDA-interfaced GPUs with high level of parallelism.

The aim of this project is to implement the Message Passing Interface for NeMo, and by doing this, improve inter-neural communication, 
to allow for better performance of parallel computations. This, in turn, will allow for bigger number of neurons within the network 
and greater level of realism brought by the simulation.

\end{abstract}

\clearpage

\chapter*{Acknowledgements}
\thispagestyle{empty}

I would like to thank Dr Timothy Todman for his guidance throughout the project and his sense of reason that helped me structure it. I would also like to thank Dr Andreas Fidjeland for his enthusiasm that sparkled my interest in this project and technical guidance that did not let me get lost at the start. I would like to thank Professor Murray Shanahan for his mathematical knowledge that helped me with clustering integration. On a separate note, I would like to thank Matthias Hueser for his technical expertise and advices.
I would also like to thank my parents for their moral support and guidance throughout this project.


\clearpage

\tableofcontents

\renewcommand{\chaptername}{}

\chapter{Introduction}

Studies in the area of neuroscience have always pushed boundaries of the development of simulation tools, requiring more computational power 
for realistical models. In order to achieve a high level of realism, large-scale networks are needed, that consist of more than \begin{math}10^8\end{math} neurons 
and \begin{math}10^{12}\end{math} synapses - and manipulating that data alone comprises a challenging task.

Consequently, parallel computations are used, in order to minimize the workload given to a particular machine and to ensure that throughout
this execution all of the clusters are communicating between each other. Once implemented, this system will allow a great amount of scalability
to be put to use, enhancing the quality of simulations.

Therefore, \emph{Message Passing Interface} (MPI) improvement in the current NeMo system will allow for faster rate of computations by making use 
of parallelism. With the use of MPI, it will be possible to make full use of cluster-based implementation that will deal with the scalability by 
distributing the workload across several machines. This means that a simulation would be able to host more neurons, therefore creating a more 
realistical model of brain activity and making it more useful for the research purposes.

However, bigger number of neurons results in need for more efficient memory management, as more data would have to be stored for effective 
communication. At the same time, a mapping specific to the topology and interconnectedness of a particular network should be derived to increase
the efficiency of transmissions.

All in all, the successful implementation of the MPI communications protocol will achieve not only a greater scope of usage of NeMo simulator within
neural computing research, but also provide a good basis for future implementations and enhancements of the simulator.


\chapter{Background}

\section{Spiking Neural Networks}

An \emph{artificial neural network}(ANN) is defined as a computational model inspired by the structure and functional aspects of biological neural networks.\cite{ActFunc}
It is an adaptive system that consists of \textbf{neurons}, its basic computational units, that are connected with each other via \textbf{synapses}.
Modern neural networks are mostly used as non-linear statistical data modeling tools to find patterns between inputs and outputs.\cite{Bar-Yam2003}

\emph{Spiking neural network}(SNN) - is a third generation neural network that comprises within itself not only concepts of neuronal and synaptic states, but 
also time.\cite{Maass1997} As a result of putting emphasis on time as an acting property, a much higher level of realism is achieved in simulations,
making SNNs particularly useful for research in such areas as brain activity or neural computation.

\subsection{Artificial and Spiking Neurons}

In order to distinguish between different generations of neural networks, one would need to take into account type of neurons, as they define the type of network.

\begin{figure}[h]
\begin{center}
\includegraphics[scale = 0.6]{images/neuron_model.png}
\end{center}
\caption{A graphical model of a simple artificial neuron\cite{Philosophy}}
\end{figure}

Typical neural networks consist of a number of computational units, \emph{artificial neurons}. An \emph{artificial neuron} is a model of a mathematical 
function or an abstraction of biological neurons\cite{Harmon1959}, that given several inputs, derives a value based on the sum of this collection of inputs 
and built-in function, and returns this value as an output, essentially acting as an axon of a real neuron. Artificial neurons differ depending on their 
specific model, such as McCulloch-Pitts or linear-threshold function. These models define a set of properties that a particular neuron possesses, such as 
transfer (or activation) function.

\begin{figure}[h]
\begin{center}
\includegraphics[scale = 0.8]{images/spiking-model.png}
\end{center}
\caption{A graphical model of a spiking neuron\cite{HidekiTanaka2009}}
\end{figure}

However, spiking neural networks are quite different in nature and, specifically, in structure of their computational units from artificial neural networks. SNNs consist of \textbf{spiking neurons} 
that aim to model the activity of real biological neurons, that is to make an abstraction which is as close as possible to the original. A \emph{spiking neuron} instead of
having set time period of emission, fires a \textbf{spike} - a very short signal that remains at its peak value for about a millisecond. Firing in spiking neurons is caused
by changes in membrane potential as well as resulting from time-based activation function. The signal is transferred to other neurons, in turn, increasing or decreasing their
membrane potential. However, since signals do not vary in value, the information is transferred via a collection of spikes, called a \emph{spike train}. Through the spike train
it is relatively easy to obtain the timing and number of spikes, therefore, these two variables hold the actual data.\cite{WulframGerstner2002} By possessing these qualities, networks 
built from spiking neurons obtain a significantly larger computational power than that of same-sized artificial neural network.\cite{Maass2003}

\subsubsection{Activation Function}

An activation function is usually an abstraction that represents the firing rate of the cell, that is dependent on the sum of inputs as well as on time.\cite{ActFunc} In spiking neural networks,
this function does not output a binary set of values, but is rather calculated through a set of differential equations that depend on the input current and, if implemented, time.

\subsubsection{Spiking Neuron Models}

Here some of the spiking (or biological) neuron models will be presented.

\begin{itemize}

\item Integrate and Fire

\emph{Integrate and Fire}(IF) model is a simple and relatively accurate representation of the actual biological neuron.
It divides the behaviour of membrane potential into two parts: long periods of \emph{integration} and short firing of \emph{spikes}. The activation function for this kind of neurons is
a time derivative of the law of capacitance, with a refactoring time added to limit the speed of firing: \begin{math}f(I) = \frac{I}{C_{m}V_{th} + t_{ref}I}\end{math}.
Therefore, the rate of firing is directly proportional to the input current.\cite{WulframGerstner2002}

However, if not modified, this model does not implement time-dependent memory. Moreover, it also lacks details in representation of biological neurons, as most of the biophysiological 
processes are simplified. For example lagging of sodium channel activation present in Hodgkin-Huxley model is absent in IF.\cite{IFModel}

\item Hodgkin and Huxley

\emph{Hodgkin and Huxley}(HH) model aims to incorporate an exhaustive description of a real biological neuron - therefore, requiring a large number of parameters to operate.
As opposed to IF, HH model uses nonlinear differential equations in its activation function to determine the membrane charasteristics and, therefore, provide the closest 
biophysical representation of a biological neuron.\cite{Hodgkin1952}

\item Izhikevich model

\emph{Izhikevich neuron model} is aiming to produce a plausible representation, close to HH, of a biological neuron, while maintaining the computational power and efficiency of an IF model.
It has several improvements over IF model, such as enhancing the activation function for the purpose of more accurate spike firing representation, which is done by introducing more 
types of spiking periods and give the implementation.\cite{Izhikevich2003}

\end{itemize}

\subsection{Synapses}

\emph{Synapses} in spiking neural networks represent dendrites in biological neural systems, making directed connections between spiking neurons. 
Therefore, synapses act as the main transmitters of spikes within the network, contributing to the connectionist approach, the main paradigm of neural networks.

\subsubsection{Synaptic Plasticity}

\emph{Synaptic plasticity} is the ability of a given synapse to change the number of receptors depending on the use.\cite{WulframGerstner2002} This represents quite an
important part of the neural network model, as it is a part of real-time alterations that occur during the actual simulation.

Spiking neural networks rely on \emph{spike timing dependent plasticity}(STDP) model for simulating plasticity, as most of information transmitted in SNNs is dependent not on 
the power of signals but rather on their timing and number. STDP rules divide the spikes occuring onto pre- and post-synaptic, determining the changes in the action potential that
they bring in. Consequently, these parameters control the extent of synaptic modification.\cite{SenSong2000}

\subsection{Topology}

Topology of a neural network is essentially its layout, that displays the connections between the neurons. Topology plays critical role, when 
the neural network is mapped onto the computational clusters, as it helps distinguish an optimal way of allocating neurons between the nodes,
and by doing this significantly increase the efficiency of communication.

\begin{figure}[h]
\begin{center}
\includegraphics[scale = 0.3]{images/topology.png}
\end{center}
\caption{Topology of a feed-forward neural network\cite{Tan2006}}
\end{figure}

\section{Spiking Neural Network Simulators}

In order to give a better outlook on the field of spiking neural network simulation, this section will cover some of the state of the art solutions present to date.

\subsection{NeMo}

\emph{NeMo} is a spiking neural network simulator aimed at real-time simulation of large-scale neuron systems with the use of highly parallel GPUs.\cite{AndreasK.Fidjeland2009}
NeMo's main purpose is to produce simulations that would be particularly useful for research, therefore, main emphasis in this tool is put onto scalability and real-time aspects
of produced simulations.

NeMo is the main platform of MPI implementation for this project. The main goal is to improve inter-neural communication between computational clusters within NeMo.
At the same time, neuron mapping should be changed to ensure efficient communication during the simulation. NeMo simulator is going to be shown in more details in the following section.

\subsection{Brian}

Another solution, \emph{Brian} aims for bigger flexibility and ease of use, therefore making it more suitable for teaching purposes. \cite{Goodman2008} This
particular simulator will provide a good example of an spiking neural network simulator, and as it is easy to operate, will be useful for learning main concepts
of SNN in detail.

\subsection{SpikeNET}

\emph{SpikeNET} is a spiking neural network simulator created for large-scale integrate-and-fire networks simulations.\cite{ArnaudDelorme1999} As the main aim of this project is to
achieve the highest possible number of neurons hosted and computed simultaneously, this solution requires a very high level of parallelism in order to operate.
Therefore, findings from this project would be useful later in the project, when the focus is going to be on the scale of computations.

\subsection{Blue Brain Project}

Initiated in cooperation between IBM and EPFL, the \emph{Blue Brain Project} aims to produce a virtual brain in a supercomputer, Blue Gene, provided by IBM.\cite{BlueBrain} Computational power
required for the operations is immense, however, supercomputing technology gave neuroscientists a set of tools to solve this problem. The Blue Brain simulator is
a very interesting project, and due to exceptionally difficult task of reducing workload across the stations - it will be particularly useful at the stage of MPI development.

\section{NeMo Architecture}




\section{Distributed Computing}

As the main objective of the project is to enhance communication efficiency between neurons in the spiking neural network, use of distributed computing plays crucial
role in its achieving. In this context, distributed computing means parallelization of tasks across the connected system and providing efficient communication medium
between separate computational clusters.

\subsection{Parallel Computing}

\emph{Parallel computing} is a form of computation where most of calculations are carried out simultaneously.\cite{G.S.Almasi1989} In order to achieve this, large problems
are divided into smaller independent parts that are carried out by separate computational units, with feeding the results either into the main cluster or holding it for further
computations. The rise of this particular field is due to the need in high-performance computing, especially in the light of frequency scaling becoming more and more
limited because of power consumption.\cite{Kumar2002}

Due to the nature of this project, a high degree of parallelism is crucial, in order to maintain the highest possible speed of simulation - the matter is that simulation would
be split between several clusters (\emph{nodes}) which, depending on the mapping, will operate independently, after receiving the information from main node at the start.

\subsection{Message Passing Interface}

\emph{Message Passing Interface} (MPI) is a language-independent protocol which acts as a communications medium for a group of processes.\cite{mpi} MPI's main function is to provide
a solid communication channel between the processes in a highly parallelised system, thus, enhancing efficiency of this system. Message passing programs are written in Fortran and 
C, with the use of built-in functions.

Currently NeMo has an MPI implementation within it, however, it is not the most optimal one. The main idea behind parallelisation in this project is that neurons are allocated, 
using mapping algorithm, between several computational nodes, which are communicating with each other with the use of MPI. Therefore, one of the main objectives will be to 
create an MPI layer inside system, that would increase the speed of computations by enhancing quality of inter-cluster communication.

There are currently several implementations of MPI to date, so here is a brief overview of available solutions.

\subsubsection{Open MPI}

The \emph{Open MPI Project} is an open-source implementation of MPI-2 maintained by a group of academic, research and industry partners, Open MPI Team.\cite{RichardL.Graham2005} This particular 
implementation aims at compatibility and high performance on all platforms, thus, making it quite easy to install and configure. Open MPI is a useful tool that will help understand 
the underlying concepts of message passing and generally give a good background within this subject area.

\subsubsection{MPICH2}

\emph{MPICH2} is an MPI implementation from Argonne National Laboratory, that aims at high performance and extendability.\cite{W.Gropp1999} The main idea of the project is to make the simulations as
fast as possible, via enhancing the efficiency of communication, therefore, MPICH2 fits well with the aims set, by providing focus on the speed and effectiveness. Another reason to choose 
MPICH2 for this project is that this package is already installed on the lab machines, therefore, saving time for the initial setup.\cite{W.Gropp1999a}

\subsection{Cluster-based approach and Mapping}

One of the most important part of the simulation is the initial \textbf{mapping} of neurons across the hosts, as it will affect the amount of inter-cluster communication and therefore overall efficiency
of the system. \emph{Mapping} defines the layout of the resulting system and is directly affected by topology - neurons are split into several groups depending on the number of synapses connecting those.
The main idea is to keep the amount of inter-cluster communication to minimum, as it is more expensive in terms of memory and time than communication within the node.

Mapping is defined by topology and hierarchy of the system - most of implementations have a master node, accountable for adding neurons and synapses to particular clusters, however, it is
possible to have a distributed system, were all nodes are equal in resposibility and actions are taken independently.

\subsubsection{Current implementations of cluster-based approach within neural networks}
At the moment there are already a few implementations present that focus on the clusterisation of the neural networks in order to achieve better inter-neuron communication and bigger throughput. 
These are the most notable ones that have a direct connection to my project.

\begin{itemize}
\item{Izhikevich's large scale model for simulating mammalian thalamocortical systems}

The work by Eugene Izhikevich, though being mostly focused on the research of brain dynamics, showed that clusterisation of the network system across several processors would result in much higher throughput rate for a large scale network - million of neurons, tens of millions of neuron compartmentss and almost half a billion synapses. In this particular case C with MPI implementation was used to allow for inter-cluster communication, and it allowed the researchers to scale the time needed for calculation of one sub-millisecond time-step for this type of network to one minute.\cite{EugeneM.Izhikevich2008}

\item{IBM cortical simulator project (C2)}

Research conducted by a group of scientists from the IBM Research Centre in Almaden focused on creation of a platform capable of simulating cerebral cortex activity. Consequently, due to extremely large scale of simulation - in this project more than 55 million Izhikevich neurons were simulated within one network - and availability of clusters within the centre, a distributed approach was applied with the use of MPI. When applied to this problem, cluster mapping managed to achieve balanced workload across the network, as well as the uniform inhibitory neuron distribution. The reason the latter point mattered in the simulation is the fact that more than 60\% of firing was inhibitory - therefore, incorrectly distributed neuron set would have a significant effect on results.\cite{DharmendraS.Modha2007}

\item{Neuromorphic model for GPGPU cluster}

Simulator produced by Bing Han and Tarek Taha aimed at the speedup of simulations with Izhikevich and Hodgkin-Huxley neuron models. Use of GPGPUs and multi-threaded MPI provided heterogeneous  clusters and per-cluster computational capacity, and clustering algorithm ensured correct distribution of neurons in accordance to the 2 layers present in the system. As a result, a significant speedup was achieved in computation, factor of 177.0 for Hodgkin-Huxley and 24.6 for Izhikevich neurons, for the task of image recognition.\cite{TarekM.Taha2010}


\chapter{Objectives and Design}

The project has a certain set of requirements that are to be satisfied, in order to accomplish it.
Here, I will provide the overview of the goals set by the project, design implementation aimed to meet those and certain properties of the problem that affected the solution.

\section{Main Objectives}

Throughout the course of project, the main aims targeted were:

\begin{enumerate}
\item {\textbf{Create a distributed version of NeMo, making it capable of scalable simulations with the use of multiple processes} (expanded in 3.2.1)}
\item {\textbf{Throughout implementation of the distributed capabilities, leave the core NeMo functionality intact - build on top} (expanded in 3.2.2)}
\item {\textbf{Implement clustering algorithm that would allow the resulting program to operate within the network capacity and distribute the workload in a more efficient manner} (expanded in 3.2.3)}
\item {\textbf{Improve the performance of large scale simulations through the use of parallel processes} (expanded in Evaluation)}
\end{enumerate}

Setting out the objectives this way helped to percieve a more detailed picture of the project as a whole and gave understanding of the set of stages that were to be accomplished in order to
make the project meet these objectives.

\section{Objective-specific Structure}

After having all of the main goals set out, it is important to get the structure of the design, with focus on satisfying these goals. In this section, the design focused on each of the features of the project will be discussed.

\subsection{Distributed Version of the Simulator}

In order to provide a more detailed picture of the distributed system, here is the set of inner classes of the NeMo core simulation classes.

\begin{figure}[h!]
\begin{center}
\includegraphics[scale = 0.6]{images/design/nemo_simulation_schematic.png}
\end{center}
\caption{Schematic representation of NeMo simulation}
\end{figure}

Early in the project design it became apparent from the experience from MPI examples\cite{}, that for the distributed simulation to run successfully, there has to be a hierarchy between classes - in this particular case, Master - Worker simulations. This way there would be a controller that would output the collected information, distribute the initial configuration and ensure synchronization between all of workers throughout timesteps.

Creation of the distributed system that would run the simulator could be split into 3 distinct steps:

\begin{enumerate}

\item{\textbf{Creation of fully separate homogeneous simulation}}

By accomplishment of this step, the system has to be able to run on several processes with little input from the user and no interaction between processes within simulation - in other words, several separate NeMo simulations with their own self-generated parameters, that could be started by user signal from the main process.

\begin{figure}[h!]
\begin{center}
\includegraphics[scale = 0.7]{images/design/design_stage_1.png}
\end{center}
\caption{Schematic model of step 1}
\end{figure}

\item{\textbf{Integrating the initialisation parameters distribution across the network}}

After thorough research conducted into the use of MPI protocol, create a simulation that would encode and pass all the initialisation parameters to the networks - mapping, configuration, neuron and synapse data - however, still without interaction between the sub-simulations.

\begin{figure}[h!]
\begin{center}
\includegraphics[scale = 0.7]{images/design/design_stage_2.png}
\end{center}
\caption{Schematic model of step 2}
\end{figure}

\item{\textbf{Implementation of the full non-constrained communication between the nodes within the network}}

Having the basic distribution system created, integrate the inter-node communication system that would allow subsystems to pass the internal spike data during simulations, while main class synchronizes the steps of all simulations by waiting for all of the communication to end, and then advancing simulation one step further.

\begin{figure}[h!]
\begin{center}
\includegraphics[scale = 0.7]{images/design/design_stage_3.png}
\end{center}
\caption{Schematic model of step 3}
\end{figure}

\end{enumerate}

Accomplishing all of these steps yields a fully working distributed implementation of the target system. It is worth noting that though this design is quite schematic, the results of the actual implementation if broken into steps were closely related to the pattern presented by this plan.

\subsection{Separation from the core NeMo System}

This section covers the structure of design that allows the distributed version be built on top of NeMo simulator without altering it.

\begin{figure}[h!]
\begin{center}
\includegraphics[scale = 0.4]{images/design/core_simulation.png}
\end{center}
\caption{NeMo core simulation design}
\end{figure}

These three classes are the core of the NeMo functionality, as they comprise all the data sets needed for the correct simulation. Therefore, in order to enable the distributed version to run separately, i.e. without interfering with the data innate to these classes, it is essential to design the system around those. In other words, the correct design must not alter internal structure of this set of classes.

In order to make the project meet this target, the structure has been developed with a strict guidance of publicly accessible properties of NeMo simulation classes. Here is the result:

\begin{figure}[h!]
\begin{center}
\includegraphics[scale = 0.5]{images/design/distributed_version_separated_design.png}
\end{center}
\caption{Distributed simulation design}
\end{figure}

As it can be easily observed, the design of the distributed version treats the core simulation classes as separate entities, only interacting with them through the NeMo-specified functionality and set of public methods, already present in them. To summarise, the MPI Layer wraps around the simulation classes and enables communication by passing input and output data between separate NeMo simulations.

\subsection{Cluster Mapping}

In this section, the structure of the mapper class will be discussed. The actual clustering function will be expressed in detail later, in Implementation.

Mapping as a procedure is taking place early during the simulation initialisation, therefore, it does not have strong time constraints. However, it is worth noting that it still has to abide the rule of non-interference within NeMo core functionality, as a result, collecting information only through the public methods, rather than dealing with the actual data sets.

Mapper class design has three main requirements it has to satisfy: correct implementation of mapping function, relative to the modularity of the network and number of processes it should be split between, compressibility, as the mapper is to be distributed among the worker simulations for the later use during communication, and quick to access, as its features would be needed for the communication.

\begin{figure}[h]
\begin{center}
\includegraphics[scale = 0.4]{images/design/mapper_design.png}
\end{center}
\caption{Mapper class design}
\end{figure}

The core functionality of the mapper requires to have a storage for neuron global-to-local and local-to-global mappings. This is accomplished by implementing two arrays that are populated during the simulation setup stage via interaction with the NeMo network class. The choice of using arrays was made as they are easy to access and relatively simple for compression, therefore meeting two goals at once.

\section{Alternative Solutions}

Throughout the stage of designing of the distributed version, there have been several possibilities of use of alternative solutions for the given task. Here, explanation of choice made at these points will be given.

\begin{itemize}

\item{\textbf{Non-hierarchical model for distributed system}}

Non-hierarchical in this context means that there would be no division master-worker between the processes - all MPI Simulation classes would be constructed in a way to support the Master capabilities - therefore achieving maximum utilisation of available resources. This approach, however, had a big overhead in terms of synchronisation, as the step function would not be as easy to implement. Moreover, the initial split of initialisation parameters would also cause an overhead in simulation - as it would have to be done inside each Worker simulation.

Therefore, the hierarchical approach was taken further into development for its distribution and synchronisation capabilities.

\item{\textbf{Single mapper, accessible across the network}}

This solution aimed at disposing of necessity to distribute the mapper across the workers. This would save space and make simulation setup time shorter, however, a lot more network capacity would be used for queries, more time would be needed to execute those and, finally, separate mapper, being a shared resource would definitely cause a bottleneck for a parallel simulation.

Therefore, after careful consideration, this idea was discarded in favor of distributing the same mapper to all workers - and by doing this increasing the speed of simulation step communication.

\item{\textbf{Provide the inner set of variables of the network to the mapper during initialisation}}

If this step is taken, less time would be needed for mapper generation - however, this would also breach the policy of non-interference with the innate NeMo properties, by altering the adding more public internal functions.

This proposal made one of the objectives unsatisfiable, thus, was not the most preferred for the aim of the project - as a result the Mapper class was constructed to interact with the public methods of Network only.

\end{itemize}

\clearpage

\section{Final Design}

\begin{figure}[h]
\begin{center}
\includegraphics[scale = 0.45]{images/design/final_design.png}
\end{center}
\caption{Final project design}
\end{figure}

By comprising all of the objective-fulfilling designs from parts 3.2.1-3.2.3 into one solution, the final design of the system was built. The diagram shows the high level design structure, full properties of each part are not shown.

It has most of NeMo core functionality and provides the distributed version of simulation that could be mapped onto any number of processes. This design satisfies all of the structural requirements set by the project and, once implemented, generates the appropriate solution to the given problem.


\chapter{Implementation}

This chapter will cover the actual implementation of the project. The whole process of implementation was split into two parts: creation of the MPI Layer and clustering implementation within the Mapper class.

\section{MPI Layer}

The MPI Layer, as mentioned before, operates, based on the two main entities - \textbf{MasterSimulation} and \textbf{WorkerSimulation} classes.

\emph{Master} is responsible for clustering and separation of the user input, distribution of this information to the corresponding workers and synchronisation of workers during simulation timesteps.

\emph{Worker} (sometimes regarded as nodes) is running the simulation and communicates with Master to receive user-input information (initialisation parameters) and with other Workers during simulation.

\subsection{Basic Distributed Platform}

The construction of the basic platform focused on establishing the communication channel with the 

\subsubsection{Initial parameters distribution}

Initial solution - basic platform + ditribution of parameters implementation

\subsection{Communication Channels Integration}

Communication scheme implementation

\subsubsection{Spike Enqueing}

Spike delivery - enqueing and updates

\subsubsection{Spike Distribution}

Spikes - firing and distribution

\section{Clustering}

Mapping algorithm implementation

\subsection{Newman's Algorithm}

Newman's work in detail

\subsection{Mapper integration}

His algorithm implemented with code snippets

\section{Alternative Solutions}

\begin{itemize}
\item{Use of Boost libraries}
\end{itemize}


\chapter{Evaluation}

After full implementation of the design proposed to the project, there has to be careful evaluation of the outcomes, or more specifically improvements introduced by such an implementation. At this point the distributed version with built-in clustering is fully developed, and the focus is on the possible criteria of the resulting product evaluation.

\section{Current Solutions}

There are already quite a few projects within the area of distributed neural network simulation. Here are some that are most similar in structure with the current project (most of them were mentioned in the Background section).

\subsection{Blue Brain Project}

The Blue Brain Project initiated in collaboration by IBM and EPFL, is aimed to reconstruct the virtual brain on the BlueGene/L supercomputer.\cite{BlueBrain} This project set its objectives in simulating the real virtual brain, as it has already succeded the simulation of a rat cortex. With the sheer numbers of neurons present in simulation closing to $10^9$, this is the project with the most resource power among listed here.

\subsection{SpikeNET}

The SpikeNET simulator focuses on the scalability of the simulation - it is concentrated on simulation of large number of integrate-and-fire neurons which spikes are computed in parallel\cite{ArnaudDelorme1999}. For most parts of its implementation SpikeNET is quite similar to the NeMo project. However, its parallel computation implementation does not exploit a fully distributed parallelism, i.e. most of computation still lies within the scope of one process.

\subsection{OpenSim}

The OpenSim project on the other hand, relies on distribution of neural simulation into a set of parallelizable sub-tasks across a cluster of machines\cite{OpenSim}. However, due to the age of its most recent implementation, the techniques used within it might already be outdated, as more efficient communication channels were developed. Nevertheless, OpenSim, due to the fact that it relies on a fully distributed system, has many similarities in structure with this project. \\ \\


Even though, all of these projects could be compared against the MPI NeMo, the differences between the implementations alone will not be sufficient to make a fully informed judgement. The problem is the lack of common ground - all of these implementations vary in the ways of neuron representation, SpikeNET lacks delay functionality and Blue Brain is built on the platform which resource base cannot be challenged by the resources of NeMo project.

Therefore, a much more sensible approach would be to compare the current version of distributed system against the "clean" NeMo version and see which particular points of simulation are different between those two.

\clearpage

\section{Criteria of Comparison}

As the test platforms are chosen, it is time to pick a set of criterias which will tested to produce informative results. In order to pick the most informative criteria, there has to be an insight into the main variables within simulation. Those are:

\begin{itemize}
\item{\textbf{Setup Time}}

This criteria comprises the temporal requirements for the simulation to be set up in order to be run. Within core NeMo functionality, this time would be comprised of the simulation setup time which is almost negligible, since the Network object constructed, operates within Simulation. For the distributed system, however, startup time is split differently: it includes the clustering time and initialisation parameters distribution window - which could be further split into 4 steps - mapper, configuration, neurons and synapses.

\item{\textbf{Simulation Time}}

The simulation time is a measure of the simulator performance according to the time spent on the actual simulation; the variables that are set for the simulations are:

\begin{enumerate}	
	\item{Number of neurons} within the network
	\item{Number of synapses per neuron}
	\item{Number of simulation steps} - constraint put on the number of simulation steps to derive time requirement per step.
\end{enumerate}

Therefore, by iterating through each of these variables, while keeping others constant, it is possible to get the dependency of the current implementation on the particular parameter, and how its value affects the overall simulation.

\item{\textbf{Network throughput}}

Last criteria that could also represent the network - total spiking within network during simulation. This value highly depends on the number of neurons and the overall interconnectivity - the current implementation would only show the number of fired neurons and external spike deliveries.

\end{itemize}

Having set out all of the meaningful criteria, it is time to look at them in detail and compare the data gathered with different systems.

\section{Setup Time}

This section will cover the simulation setup stage of the distributed system's initialisation.

\subsubsection{MPI clustering and distribution overhead}

\begin{figure}[h]
\begin{center}
\includegraphics[scale = 0.5]{images/setup_comparison.png}
\end{center}
\caption{Comparison of setup time to the number of neurons, synapses and processes}
\end{figure}

From the above graph it is easily observable that the most amplifying factor for simulation setup time is the number of neurons. The setup time, however, can be further split into the 5 times - clustering, mapper, configuration, neuronal and synaptic distributions.

The ones having the most effect on the simulation setup time are clustering and synaptic distribution. Now, if those are observed closer under particular circumstances, their dependance upon other variables is clearly seen:

\begin{figure}[ht]
\begin{minipage}[b]{0.5\linewidth}
\centering
\includegraphics[width=\textwidth]{images/syndistr.png}
\caption{Synaptic distribution}
\label{fig:figure1}
\end{minipage}
\hspace{0.5cm}
\begin{minipage}[b]{0.5\linewidth}
\centering
\includegraphics[width=\textwidth]{images/cluster.png}
\caption{Clustering}
\label{fig:figure2}
\end{minipage}
\end{figure}

From these graphs, it can be concluded that synaptic distribution depends more on the number of neurons and workers rather than on the per-neuron synapse count. As for the clustering, its timeline correlates with the increase in the number of processes - which is logically correct, since this means more calls to the clustering function at the startup.

\section{Simulation Time}

This section attempts to collect and evaluate the results of multiple distributed simulations against the original NeMo simulation, and show the particular differences through performance.

\subsection{Number of neurons}

Graphs and explain how the evaluated trends on the graphs show efficiency of the algorithm against number of neurons

\subsection{Number of synapses}

Same - but now with varying neurons (need mapping implemented)s

\subsection{Number of steps}

Runtime of the simulation in focus, no of steps constant

\section{Network throughput during simulation}

Firing and spike delivery through the network - how does it compare.


\chapter{Conclusion}

The work that was produced within the scope of this project clearly shows that distributed implementation can yield a much better performance time, when applied to large-scale networks.

\section{Outcome of the project}

Is the distributed version successful in reaching targets set by the project?

\section{Possible impact}

How this project might impact further developments within the field?

\section{Further work}

Possible extensions of the project

\subsection{Short-term extensions}

\subsubsection{Optimisation}

Even though, one of the potential aims of this project is to provide better performance in simulations, most of the benchmarks proved distributed version to be slower than the single machine one. The main reasons behind this are: simulation setup overhead and communication overhead - and in the latter case, the main reason is ineffective access of the data fields.

\subsubsection{Implement full NeMo functionality}

The distributed version provides most of the functionality needed, however, not every single function was implemented. The main focus is on the STDP application and backend setup per machine.

STDP application is failed due to inability to return full STDP function via encoding. This is resolved through extra message passing.

Backend setup needs to be implemented via NeMo functionality - whenever the machine, the process runs on has CUDA capabilities, the worker should have an indication sent to the Master node.

\subsection{Long-term extensions}

Possible developments to extend the project into new projects within this field


\clearpage

\bibliographystyle{plain}
\bibliography{myrefs}

\begin{appendices}

\chapter{Simulation Classes}
\section{MasterSimulation}
\lstinputlisting{code/MasterSimulation.hpp}
\clearpage
%This page is intentionally left blank
\lstinputlisting{code/MasterSimulation.cpp}
\clearpage
\section{WorkerSimulation}
\lstinputlisting{code/WorkerSimulation.hpp}
\clearpage
%This page is intentionally left blank
\lstinputlisting{code/WorkerSimulation.cpp}
\clearpage
\chapter{Full Mapper Implementation}
\lstinputlisting{code/MapperSim.hpp}
\clearpage
%This page is intentionally left blank
\lstinputlisting{code/MapperSim.cpp}
\clearpage

\end{appendices}

\end{document}
