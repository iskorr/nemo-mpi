\chapter{Objectives and Design}

This chapter will provide the overview of the goals set by the project, design implementation aimed to meet those and certain properties of the problem that affected the solution.
It will also cover design paradigms used for project guidance, as well as, the set of tools used for outlining its aspects.

\section{Main Objectives}

Throughout the course of project, the main aims that were targeted had been:

\begin{itemize}
\item {\textbf{Create a distributed version of NeMo, making it capable of scalable simulations with the use of multiple processes}}
\item {\textbf{Lower the requirements set by the platform for the target machine(s) through parallelization}}
\item {\textbf{Implement clustering algorithm that would allow the resulting program to operate within the network capacity and distribute the workload in a more efficient manner}}
\item {\textbf{Throughout implementation of the distributed capabilities, leave the core NeMo functionality intact - build on top}}
\end{itemize}

Setting out the objectives this way helped to percieve a more detailed picture of the project as a whole and gave understanding of the set of stages that were to be accomplished in order to
make the project meet these objectives. Here are the steps, that I divided the project into, while having the goals and the initial plan as a guideline:

\begin{itemize}
\item {\textbf{Create a distributed version of NeMo, making it capable of scalable simulations with the use of multiple processes}}
\item {\textbf{Lower the requirements set by the platform for the target machine(s) through parallelization}}
\item {\textbf{Implement clustering algorithm that would allow the resulting program to operate within the network capacity and distribute the workload in a more efficient manner}}
\item {\textbf{Throughout implementation of the distributed capabilities, leave the core NeMo functionality intact - build on top}}
\end{itemize}

\section{Objective-specific Structure}

After having all of the main goals set out, it is important to get the structure of the design 

\subsection{Design Criteria}

\subsection{Initial Proposal}

\subsection{Alternative Solutions}
\begin{itemize}

\item {Latency and Spike delivery}

Due to latency within the network of processors the system was tested on, there were a few precedents of race conditions that caused uncertain behaviour of the simulator - therefore, giving output that differed significantly from the desired one. In order to overcome this drawback, a timer system within the master simulation has been implemented - this ensured control of the communications, and allowed for stops with delay, to deal with propagating spikes. That move has significantly increased precision of the simulation, moreover, giving a better overview of the performance.

\item {}

\end{itemize}


\section{Final Design}

\section{Tools}
