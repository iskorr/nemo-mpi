\chapter{Implementation}

This chapter will cover the actual implementation of the project. The whole process of implementation was split into two parts: creation of the MPI Layer and clustering implementation within the Mapper class.

\section{MPI Layer}

The MPI Layer, as mentioned before, operates, based on the two main entities - \textbf{MasterSimulation} and \textbf{WorkerSimulation} classes.

\emph{Master} is responsible for clustering and separation of the user input, distribution of this information to the corresponding workers and synchronisation of workers during simulation timesteps.

\emph{Worker} (sometimes regarded as nodes) is running the simulation and communicates with Master to receive user-input information (initialisation parameters) and with other Workers during simulation.

\subsection{Basic Distributed Platform}

The construction of the basic platform focused on establishing the communication channel with the 

\subsubsection{Initial parameters distribution}

Initial solution - basic platform + ditribution of parameters implementation

\subsection{Communication Channels Integration}

Communication scheme implementation

\subsubsection{Spike Enqueing}

Spike delivery - enqueing and updates

\subsubsection{Spike Distribution}

Spikes - firing and distribution

\section{Clustering}

Mapping algorithm implementation

\subsection{Newman's Algorithm}

Newman's work in detail

\subsection{Mapper integration}

His algorithm implemented with code snippets

\section{Alternative Solutions}

\begin{itemize}
\item{Use of Boost libraries}
\end{itemize}
